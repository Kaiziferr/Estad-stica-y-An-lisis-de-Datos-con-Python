\documentclass{article}
\usepackage[utf8]{inputenc}

\title{ApuntesEstadistica}
\author{Steven Bernal }


\begin{document}

\maketitle

\section{Estadistica}
Ciencia que recoge, organiza, presenta, analiza e interpreta datos con el fin de propiciar una toma de decisiones en ambientes de incertidumbre.

\subsection{Tipos de estadística}

\textbf{Estadística descriptiva}:
métodos para organizar, resumir y presentar datos de manera informativa. Donde se construyen indicadores, se hacen gráficos, se realizan comparaciones, siempre con el interés de conocer sobre la población de donde fue tomada la muestra.\\

El interés en la Estadística Descriptiva se centra en tres elementos:

\begin{itemize}
	
    \item El análisis de patrones o tendencias
    \item La medición de la variación
    \item La identificación de la forma o distribución de los datos.
    
\end{itemize}
		La finalidad de la Estadística descriptiva NO es extraer conclusiones sobre el fenómeno de estudio, sino solamente su descripción.
	
\paragraph{Estadistica Inferencial:}
	métodos que se emplean para determinar una propiedad de una población con base en la información de una muestra de ella.
	
\begin{itemize}
    \item El muestreo
    \item Las distribuciones de probabilidad
    \item La estimación de parámetros
    \item La prueba de hipótesis
    \item Análisis de correlación
    \item  Los modelos estadísticos de regresión
    \item El diseño de experimentos
\end{itemize}

\vspace{5 mm}
La aplicación del tratamiento estadístico tiene dos fases fundamentales:

\begin{itemize}
    \item Organización y análisis inicial de los datos recogidos (Descriptiva)
    \item Extracción de conclusiones válidas de los datos recogidos (Inferencial)
\end{itemize}

 \subsection{Datos}\\
    \textbf{Obtención de datos}\\
La materia prima de la estadística son los datos, los cuales son el resultado de la observación de alguna(s) característica(s) de los elementos de interés en cierto estudio (Variables). La naturaleza de la característica y el instrumento que se dispone para registrar la misma, definirá el tipo de escala de medición que se ajusta.\\

    \textbf{Tipos de datos}
    \begin{itemize}
        \item \textbf{Datos primarios:} Son aquellos que el investigador obtiene directamente de la realidad, recolectándolos con sus propios instrumentos. Los datos primarios son escritos durante el tiempo que se está estudiando o por la persona directamente envuelta en el evento
        \item \textbf{Datos secundarios:}
        Registros escritos que proceden también de un contacto con la práctica, pero que ya han sido recogidos y muchas veces procesados por otros investigadores.
    \end{itemize}
    
    \textbf{Muestra bruta}
    Muestra que no ha tenido ningún tipo de tratamiento.

\section{Estad́ıstica descriptiva}
\paragraph{Poblacion:}
conjunto de elementos que comparten característica de interes para un estudio y hacia los cuales se extenderán las conclusiones.

\paragraph{Muestra:}
porción o parte de la población de interés, la cual debe ser representativa. El tamaño de la muestra está relacionado con el grado de homogeneidad (variabilidad de la característica de interés) de la población, así como del nivel de precisión requerido.

\paragraph{Parámetro:}
es un valor, medida o indicador representativo de la población que se selecciona para ser estudiado

\paragraph{Estadístico:}
es el elemento que describe una muestra y sirve como una aproximación del parámetro de la población correspondiente

\paragraph{Medición:}
proceso por el cual asignamos un valor a una variable a determinada unidad de análisis.

\paragraph{Variable:}
característica medible de la población, la cual es de interés.

\subsection{Tipos de variables}
\begin{itemize}
    \item \textbf{Variable cualitativa}\\
    Característica que se estudia  de naturaleza no numérica.  Algunos ejemplos de variables cualitativas son el género,  la filiación religiosa, tipo de automóvil que se posee, estado de nacimiento y color de ojos.
    \item \textbf{Variable cuantitativa}\\
    Variable de caracter numérica y poseen un orden inherente. Algunos ejemplos de variable cuantitativo son la edad, el saldo de cuenta, peso y los numeros de hijos.\\
    
        Las variables cuantitativas pueden ser discretas o continuas
    
    \begin{itemize}

        \item \textbf{Variables continuas}  toman cualquier valor dentro de un intervalo específico.
        \item \textbf{Variables discretas:} adoptan sólo ciertos valores y existen vacíos entre ellos.
    \end{itemize}
    
\end{itemize}
    
    \subsection{Escalas de Medición}
    \begin{itemize}
        \item \textbf{Escala nominal}\\
        	Las observaciones acerca de una variable cualitativa sólo se clasifican y se cuentan.  No existe una forma particular para ordenar las etiquetas. En el nivel nominal, la medición consiste en contar. A veces, para una mejor comprensión de lectura, estos conteos se convierten en porcentajes.
        \item\textbf{Escala ordinal}\\
        Cuando los datos muestran las propiedades de los datos nominales y además tienen sentido el orden o jerarquía de los datos.
        
        \item \textbf{Escala de intervalo}\\
        Las características del nivel ordinal, pero, además, la diferencia entre valores constituye una magnitud constante. Es importante destacar que 0 es un punto más en la escala. No representa la ausencia de estado.  El 0 significa auscencia de la propiedad mas no, de la caracteristica. Ejemplo: Temperatura: 0 no significa ausencia de temperatura.
        
        \item \textbf{Escala de razon}
        	Los datos cuantitativos son registrados en el nivel de razón de la medición. El nivel de razón es el más alto. Posee todas las características del nivel de intervalo, aunque, además, el punto 0 tiene sentido y la razón entre dos  números es significativa. Ejemplo: Dinero, peso, velocidad.
    \end{itemize}
    
   
\end{document}
